\begin{quotation}
	\textit{\textquote{Ma messe est liturgique ... En mettant le Credo en musique, j'ai seulement voulu préserver le texte d'une manière particulière. On compose une marche pour aider les hommes à marcher ; ainsi, avec mon Credo, j'espère fournir une aide pour le texte.}}
\end{quotation}

C'est ainsi que s'exprimait Igor Stravinsky, un compositeur du XX\textsuperscript{e} siècle, au sujet d'une de ses messes. Cette réflexion du compositeur russe, bien qu'éloignée chronologiquement de la période baroque, cristallise une vérité fondamentale sur la fonction de la musique : celle-ci ne se contente pas d'exister en tant qu'art autonome, elle accompagne, soutient et révèle d'autres dimensions de l'expérience humaine. La musique devient ainsi un vecteur d'accès, une \textquote{aide} comme le formule Stravinsky, qui rend perceptible et accessible ce qui pourrait demeurer abstrait.

L'analogie que propose Stravinsky entre la composition musicale et la marche révèle une dimension pragmatique de l'art musical souvent occultée par les approches purement esthétiques. La mise en musique du texte sacré offre un support, un cadre qui guide la réception et la mémorisation. Cette fonction d'accompagnement, d'aide à la compréhension et à la transmission, traverse toute l'histoire de la musique occidentale. Elle se manifeste avec une acuité particulière dans la musique baroque française, où l'alliance du texte et de la mélodie, de la danse et du chant, de la cérémonie et de l'art, répond à des impératifs tant esthétiques que fonctionnels.

Cette conception fonctionnelle de la musique comme medium d'\gls{accessibilite} trouve un écho particulièrement pertinent dans les enjeux contemporains de la valorisation du patrimoine musical. À l'ère numérique, les institutions patrimoniales sont confrontées à un défi analogue : comment composer, non plus une marche ou un Credo, mais des outils numériques qui aident les chercheurs à chercher, les musiciens à interpréter, les enseignants à transmettre ? Comment préserver et transmettre l'immense corpus de la musique ancienne afin d'en faciliter l'accès et la compréhension ?

En effet, la révolution numérique a profondément transformé les modalités de conservation, d'étude et de diffusion du patrimoine musical. Les institutions culturelles disposent aujourd'hui de moyens techniques sans précédent pour numériser, cataloguer et mettre à disposition des corpus considérables. Cette démocratisation potentielle de l'accès aux sources musicales anciennes s'accompagne toutefois de nouveaux défis. L'accumulation exponentielle de données numériques pose des questions inédites en termes d'organisation, de structuration et surtout d'\gls{accessibilite} effective. La simple mise en ligne de documents numérisés ne suffit plus ; elle doit être accompagnée de dispositifs de \glslink{mediationculturelle}{médiation}, d'outils de recherche sophistiqués et d'interfaces adaptées aux différents publics. L'enjeu ne porte plus seulement sur la conservation des œuvres, mais sur leur capacité à demeurer vivantes, utilisables et signifiantes pour les communautés scientifiques, artistiques et éducatives contemporaines.

Cette problématique revêt une dimension particulièrement aigüe pour les musiques anciennes, dont la transmission repose sur des chaînes documentaires complexes. Entre les sources manuscrites originales, souvent lacunaires ou dispersées, les éditions critiques modernes et les interprétations contemporaines, se dessine un écosystème informationnel foisonnant mais fragmenté. Les bases de données patrimoniales tentent de cartographier ce territoire en reliant sources, œuvres, auteurs, interprètes et contextes historiques. Elles deviennent ainsi les nouveaux lieux de mémoire de la musique ancienne, mais leur efficacité dépend entièrement de leur capacité à être appropriées par leurs utilisateurs.

Le \gls{cmbv}, institution de référence pour la recherche, la diffusion et la valorisation du patrimoine musical français des XVII\textsuperscript{e} et XVIII\textsuperscript{e} siècles, s'inscrit pleinement dans cette logique d'accompagnement. Depuis sa création, il s'attache à construire des ponts entre la recherche académique, la pratique musicale vivante et la \gls{mediationculturelle}. Ses bases de données, fruit de décennies de travaux scientifiques, constituent un corpus exceptionnel recensant compositeurs, œuvres, sources, événements musicaux et interprètes de la période baroque française.

Toutefois, l'accumulation de ces données soulève aujourd'hui des questions cruciales d'\gls{accessibilite}. Comment garantir que ces ressources, issues de recherches approfondies, demeurent utilisables et pertinentes pour des publics aux besoins différenciés ? Comment concilier exhaustivité scientifique et facilité d'usage ? Comment les contraintes techniques et budgétaires contemporaines peuvent-elles orienter vers des solutions innovantes et durables ?

La base \textit{Philidor~4} du \gls{cmbv} cristallise ces enjeux. Conçue selon une architecture modulaire sophistiquée, elle offre un accès structuré à un corpus considérable tout en révélant certaines limites en termes d'ergonomie et de navigation. Face à ces défis, l'\gls{edition-numerique} statique émerge comme une solution complémentaire prometteuse, capable de transformer les données brutes en parcours éditorialisés, plus accessibles et contextualisés.

Ce mémoire se propose d'analyser cette tension entre accumulation patrimoniale et \gls{accessibilite} numérique, en étudiant comment une approche éditoriale peut constituer une réponse soutenable aux enjeux contemporains de valorisation du patrimoine musical numérique. La problématique centrale peut être formulée ainsi : dans quelle mesure l'\gls{edition-numerique} statique peut-elle offrir une solution complémentaire et soutenable aux limites d'\gls{accessibilite} des bases de données patrimoniales traditionnelles ?

Cette interrogation se décline selon plusieurs axes d'analyse. D'abord, il s'agira de comprendre les mutations qui traversent aujourd'hui les institutions patrimoniales face aux défis du numérique, en prenant l'exemple du \gls{cmbv} et de son évolution depuis trente ans. Puis, nous ferons un tour d'horizon des systèmes ayant été ou étant utilisés pour gérer, sauvegarder et diffuser les données du \gls{cmbv}. Ensuite, nous analyserons les potentiels et les limites de la base \textit{Philidor~4}, tant du point de vue technique qu'ergonomique, pour identifier les enjeux d'\gls{accessibilite} qu'elle soulève. Enfin, nous explorerons les possibilités offertes par l'\gls{edition-numerique} statique, en développant un prototype de site web thématique généré à partir des données de \textit{Philidor}.

La démarche adoptée articule analyse institutionnelle, étude technique et expérimentation pratique. Elle s'appuie sur une approche qualitative, combinant exploration des archives des projets numériques, analyse des outils existants et développement d'une solution prototype. Cette méthodologie permet d'ancrer la réflexion théorique dans une expérience concrète de manipulation des données et de création d'interfaces.

À travers l'exemple du \gls{cmbv} et de sa base \textit{Philidor~4}, nous explorerons ainsi les mutations institutionnelles, techniques et conceptuelles qui redéfinissent aujourd'hui les modalités d'accès aux corpus musicaux anciens, questionnant les formes contemporaines de cette \textquote{aide} que la technologie peut apporter à la transmission du patrimoine.