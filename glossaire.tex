% Définition des acronymes
\newacronym{cmbv}{CMBV}{Centre de musique baroque de Versailles}
\newacronym{tnah}{TNAH}{Technologies Numériques Appliquées à l'Histoire}
\newacronym{sgbd}{SGBD}{Système de Gestion de Base de Données}
\newacronym{xml}{XML}{\textit{eXtensible Markup Language}}
\newacronym{xslt}{XSLT}{\textit{eXtensible Stylesheet Language Transformations}}
\newacronym{rgpd}{RGPD}{Règlement général sur la protection des données}
\newacronym{sigb}{SIGB}{Système Intégré de Gestion de Bibliothèque}
\newacronym{cms}{CMS}{\textit{Content Management System}}
\newacronym{saas}{SAAS}{\textit{Software as a Service}}
\newacronym{iccare}{ICCARE}{Industries culturelles et créatives : action, recherche expérimentation}
\newacronym{skos}{SKOS}{\textit{Simple Knowledge Organization System}}
\newacronym{sql}{SQL}{\textit{Structured Query Language}}
\newacronym{iiif}{IIIF}{\textit{International Image Interoperability Framework}}
\newacronym{api}{API}{\textit{Application Programming Interface}}
\newacronym{rdf}{RDF}{\textit{Resource Description Framework}}
\newacronym{owl}{OWL}{\textit{Ontology Web Language}}
\newacronym{ocr}{OCR}{\textit{Optical Character Recognition}}
\newacronym{omr}{OMR}{\textit{Optical Music Recognition}}
\newacronym{wysiwyg}{WYSIWYG}{\textit{What you see is what you get}}
\newacronym{html}{HTML}{\textit{HyperText Markup Language}}
\newacronym{uuid}{UUID}{\textit{Universally unique identifier}}
\newacronym{nlp}{NLP}{\textit{Natural Language Processing}}
\newacronym{frbr}{FRBR}{\textit{Functional Requirements for Bibliographic Records}}
\newacronym{cnrs}{CNRS}{Centre national de la recherche scientifique}
\newacronym{ircam}{IRCAM}{Institut de recherche et coordination acoustique/musique}
\newacronym{cesr}{CESR}{Centre d'études supérieures de la Renaissance}
\newacronym{iremus}{IReMus}{Institut de Recherche en Musicologie}
\newacronym{rism}{RISM}{Répertoire international des sources musicales}
\newacronym{bnf}{BnF}{Bibliothèque nationale de France}
\newacronym{viaf}{VIAF}{\textit{Virtual International Authority File}}
\newacronym{idref}{IDREF}{Identifiants et référentiels pour l'Enseignement supérieur et la recherche}
\newacronym{isni}{ISNI}{\textit{International Standard Name Identifier}}
\newacronym{crr}{CRR}{Conservatoire à rayonnement régional}
\newacronym{cnsmd}{CSMND}{Conservatoires nationaux supérieurs de musiques et de danses}
\newacronym{anr}{ANR}{Agence Nationale de la Recherche}
\newacronym{tei}{TEI}{\textit{Text Encoding Initiative}}
\newacronym{alto}{ALTO}{\textit{Analysed Layout and Text Object}}
\newacronym{url}{URL}{\textit{Uniform Resource Locator}}
\newacronym{ark}{ARK}{\textit{Archival Resource Key}}
\newacronym{jsonld}{JSON-LD}{\textit{JavaScript Object Notation for Linked Data}}
\newacronym{csv}{CSV}{\textit{Comma-separated values}}
\newacronym{uri}{URI}{\textit{Uniform Resource Identifier}}
\newacronym{w3c}{W3C}{\textit{World Wide Web Consortium}}
\newacronym{unimarc}{UNIMARC}{\textit{UNIversal Machine-Readable Cataloguing}}
\newacronym{sudoc}{SUDOC}{ Système universitaire de documentation}
\newacronym{pdf}{PDF}{\textit{Portable Document Format}}
\newacronym{rtf}{RTF}{\textit{Rich Text Format}}
\newacronym{xss}{XSS}{scripts inter-sites}

% Définition des termes du glossaire
\newglossaryentry{edition-numerique}{
	name={édition numérique},
	description={Processus de création, structuration et diffusion de contenus textuels ou multimédias via des supports numériques}
}

\newglossaryentry{editorialisation-num}{
	name={éditorialisation numérique},
	description={Toutes les actions destinées à structurer, rendre accessible et visible un contenu sur le web}
}

\newglossaryentry{accessibilite}{
	name={accessibilité},
	description={Qualité d'un système permettant son utilisation par le plus grand nombre d'utilisateurs}
}

\newglossaryentry{interoperabilite}{
	name={interopérabilité},
	description={Capacité de différents systèmes informatiques à fonctionner ensemble et à échanger des données}
}

\newglossaryentry{prosopographie}{
	name={prosopographie},
	description={Etude collective qui cherche à dégager les caractères communs d’un groupe d’acteurs historiques en se fondant sur l’observation systématique de leurs vies et de leurs parcours.}
}

\newglossaryentry{web-semantique}{
	name={web sémantique},
	description={Evolution du Web visant à structurer et enrichir les données de manière à ce que les utilisateurs puissent trouver, partager et combiner l’information plus facilement, sans avoir besoin d’intermédiaires.}
}

\newglossaryentry{continuiste}{
	name={continuiste},
	description={Musicien, souvent un organiste ou un claveciniste, qui guide subtilement l’ensemble, faisant vivre la musique et soutenant les solistes, tout en laissant une grande liberté expressive.},
	plural={contunistes}
}

\newglossaryentry{scalabilité}{
	name={scalabilité},
	description={Capacité d’un système, d’une organisation ou d’un dispositif à soutenir une croissance forte et rapide, en maintenant ses performances et en optimisant l’utilisation des ressources, que ce soit en termes de coûts, d’efficacité ou de qualité de service.},
}

\newglossaryentry{mediationculturelle}{
    name={médiation culturelle},
    description={Ensemble des pratiques visant à mettre en relation les publics avec les œuvres, les savoirs ou les patrimoines, afin de favoriser la compréhension, l’appropriation et l’accès à la culture.},
}

\newglossaryentry{ancienregime}{
    name={ancien régime},
    description={Période historique précédant la Révolution française, caractérisée par la monarchie absolue, la société d’ordres et un système économique et social fondé sur les privilèges. Par extension, désigne un ordre ancien ou un système social ou politique révolu.},
}

\newglossaryentry{institutionnalisation}{
    name={institutionnalisation},
    description={Processus par lequel une pratique, une norme ou une organisation acquiert un caractère officiel et reconnu, en s’intégrant durablement dans des structures sociales, culturelles ou juridiques.},
}

\newglossaryentry{ontologie}{
    name={ontologie},
    description={En informatique et en sciences de l’information, modèle conceptuel formalisé qui décrit un domaine de connaissances en définissant les entités, leurs propriétés et leurs relations.},
    plural={ontologies}
}

\newglossaryentry{taxonomie}{
    name={taxonomie},
    description={Méthode de classification hiérarchique des éléments, concepts ou objets selon des critères définis, utilisée notamment en biologie, en sciences de l’information et en organisation des connaissances.},
    plural={taxonomies}
}

\newglossaryentry{workflow}{
    name={workflow},
    description={Ensemble structuré d’étapes ou de tâches nécessaires à la réalisation d’un processus, souvent automatisé dans un environnement numérique pour améliorer l’efficacité et la coordination.},
    plural={workflows}
}

\newglossaryentry{incipitmusical}{
    name={incipit musical},
    description={Début d’une œuvre musicale, souvent utilisé comme élément d’identification dans les catalogues, les bases de données ou les éditions critiques.},
    plural={incipits musicaux}
}

\newglossaryentry{thesaurus}{
    name={thésaurus},
    description={Outil d’organisation des connaissances consistant en une liste structurée et hiérarchisée de termes, reliés entre eux par des relations sémantiques, pour faciliter l’indexation et la recherche d’information.},
    plural={thésauri}
}

\newglossaryentry{bibliotheconomie}{
    name={bibliothéconomie},
    description={Discipline qui étudie l’organisation, la gestion et le fonctionnement des bibliothèques, ainsi que les méthodes de traitement, de conservation et de diffusion des documents.},
}

\newglossaryentry{algorithmie}{
    name={algorithmie},
    description={Discipline qui s’intéresse à la conception, à l’analyse et à la mise en œuvre d’algorithmes, c’est-à-dire de suites d’instructions permettant de résoudre un problème ou d’accomplir une tâche.},
}

\newglossaryentry{editionwebstatique}{
    name={édition web statique},
    description={Méthode de publication en ligne reposant sur des fichiers statiques, sans génération dynamique côté serveur, permettant de réduire la consommation énergétique et d’améliorer la sécurité et la rapidité d’affichage.},
}

\newglossaryentry{callback}{
    name={callback},
    description={Mécanisme de programmation où une fonction est passée en argument à une autre et appelée ultérieurement, généralement pour gérer des événements ou des opérations asynchrones.},
    plural={callbacks}
}

\newglossaryentry{serialisation}{
    name={sérialisation},
    description={Processus de transformation d’un objet ou d’une structure de données en une séquence de caractères ou d’octets, afin de pouvoir la stocker, la transmettre ou la reconstruire ultérieurement.},
}